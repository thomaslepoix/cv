\def\debug{0} % [0;1] Show all boxes
\def\black{0} % [0;1]

\documentclass[a4paper]{article}
%\documentclass{standalone}

\pagestyle{empty} % Disable headers and footers
\setlength{\parindent}{0pt} % Disable paragraph indentation

\usepackage{fontspec}
\setmainfont[Ligatures=TeX, Mapping=tex-text, SmallCapsFont = [Ubuntu-C.ttf]]{Ubuntu}
\usepackage{fontawesome}
\usepackage[
	lmargin=0mm,
	rmargin=0mm,
	tmargin=0mm,
	bmargin=0mm,
	layout=a4paper
	]{geometry}
\usepackage{xcolor}
\usepackage{tikz}
\usepackage{hyperref}
\hypersetup{
	pdfpagemode={UseOutlines},
	bookmarksopen=true,
	bookmarksopenlevel=0,
	hypertexnames=false,
	colorlinks=true,
	linkcolor=black,
	urlcolor=black,
	pdfstartview={FitV},
	unicode,
	breaklinks=true,
	}

\begin{document}

\setlength{\fboxrule}{1.1pt}
%\setlength{\fboxsep}{0pt}
\centering

\ \vspace{0.5cm}%%%%%%%%%%%%%%%%%%%%%%%%%%%%%%%%%%%%%%%%%%%%%%%%%%%%%%%%%%HEADER

\if\black1\colorbox{black}{ \fi\begin{minipage}[H]{0.75\paperwidth}
	\centering
	\if\black0\rule{\textwidth}{1.1pt}\\\fi
	\if\black1\vspace{0.3cm}\fi
	\if\black1\color{white}\fi
	\huge THOMAS LEPOIX\\
	\LARGE Ingénieur systèmes embarqués\\
	\normalsize
	\if\black1\vspace{0.3cm}\fi
	\if\black0\rule{\textwidth}{1.1pt}\\\fi
	\end{minipage}\if\black1} \fi

\vspace{0.5cm}%%%%%%%%%%%%%%%%%%%%%%%%%%%%%%%%%%%%%%%%%%%%%%%%%%%%%%%COORDONNÉES

\if\debug1\fbox{\fi\begin{minipage}[t]{0.35\paperwidth}
	\centering
	%\fbox{\textbf{COORDONNÉES}}\\
	\fbox{\parbox[c][10pt][b]{77pt}{\centering\textbf{COORDONNÉES}}}\\
	\vspace{0.5cm}
	\if\debug1\fbox{\fi\begin{minipage}[H]{0.25\paperwidth}
		\raggedleft
		27 rue Gensan 33000 Bordeaux\\
		thomas.lepoix@protonmail.ch\\
		0683604789\\
		\href{https://github.com/thomaslepoix}{thomaslepoix}\\
		\href{https://www.linkedin.com/in/thomas-lepoix-298758150/}{thomas-lepoix-298758150}\\
		Permis B\\
		17/01/1996\\
		\end{minipage}\if\debug1} \fi
%		\end{minipage}
	\hspace{0.25cm}
	\if\debug1\fbox{\fi\begin{minipage}[H]{0.02\paperwidth}
		\centering
		\faicon{map-marker}\\
		\faicon{at}\\
		\faicon{phone}\\
		\href{https://github.com/thomaslepoix}{\faicon{github}}\\
		\href{https://www.linkedin.com/in/thomas-lepoix-298758150/}{\faicon{linkedin}}\\
		\faicon{car}\\
		\faicon{birthday-cake}\\
		\end{minipage}\if\debug1} \fi
	\if\debug1\fbox{\fi\begin{minipage}[H]{0.02\paperwidth}
		\ %alignement box
		\end{minipage}\if\debug1} \fi

	\vspace{0.5cm}%%%%%%%%%%%%%%%%%%%%%%%%%%%%%%%%%%%%%%%%%%%%%%%%%%%COMPÉTENCES
	%\fbox{\textbf{COMPÉTENCES}}\\
	\fbox{\parbox[c][10pt][b]{74pt}{\centering\textbf{COMPÉTENCES}}}\\
	\vspace{0.5cm}
%	\if\debug1\fbox{\fi\begin{minipage}[H]{0.33\paperwidth}
	\if\debug1\fbox{\fi\begin{minipage}[H]{0.32\paperwidth}
		\raggedleft
%%		Conception électronique \hspace{0.2cm} \faicon{circle} \faicon{circle} \faicon{circle-thin}\\
%		Électronique (BF, HF, CEM) \hspace{0.2cm} \faicon{circle} \faicon{circle} \faicon{circle}\\
%%		Hyperfréquences \hspace{0.2cm} \faicon{circle} \faicon{circle} \faicon{circle}\\
%		Linux embarqué (Yocto) \hspace{0.2cm} \faicon{circle} \faicon{circle} \faicon{circle}\\
%		Environnement libre / GNU \hspace{0.2cm} \faicon{circle} \faicon{circle} \faicon{circle}\\
%		Microcontrôleurs (8 bits / ARM) \hspace{0.2cm} \faicon{circle} \faicon{circle-thin} \faicon{circle-thin}\\
%		Bus (UART, SPI, I2C, CAN, etc.) \hspace{0.2cm} \faicon{circle} \faicon{circle} \faicon{circle-thin}\\
%		Réseau / infra \hspace{0.2cm} \faicon{circle} \faicon{circle-thin} \faicon{circle-thin}\\
%		DevOps (CMake, Docker, Debian) \hspace{0.2cm} \faicon{circle} \faicon{circle} \faicon{circle-thin}\\
%		C / C++ \hspace{0.2cm} \faicon{circle} \faicon{circle} \faicon{circle}\\
%		Assembleur CISC Freescale \hspace{0.2cm} \faicon{circle} \faicon{circle-thin} \faicon{circle-thin}\\
%		Python \hspace{0.2cm} \faicon{circle} \faicon{circle-thin} \faicon{circle-thin}\\
%		Scripts shell \hspace{0.2cm} \faicon{circle} \faicon{circle} \faicon{circle}\\
%		LaTex \hspace{0.2cm} \faicon{circle} \faicon{circle} \faicon{circle}\\
%		Anglais \hspace{0.2cm} \faicon{circle} \faicon{circle} \faicon{circle-thin}\\
		Linux embarqué (Yocto) \hspace{0.2cm} \faicon{circle} \faicon{circle} \faicon{circle}\\
		C / C++ \hspace{0.2cm} \faicon{circle} \faicon{circle} \faicon{circle}\\
		Électronique (BF, HF, CEM) \hspace{0.2cm} \faicon{circle} \faicon{circle} \faicon{circle}\\
		Environnement libre / GNU \hspace{0.2cm} \faicon{circle} \faicon{circle} \faicon{circle}\\
		Scripts shell \hspace{0.2cm} \faicon{circle} \faicon{circle} \faicon{circle}\\
		LaTex \hspace{0.2cm} \faicon{circle} \faicon{circle} \faicon{circle}\\
		DevOps (CMake, Docker, Debian) \hspace{0.2cm} \faicon{circle} \faicon{circle} \faicon{circle-thin}\\
		Bus (UART, SPI, I2C, CAN, etc.) \hspace{0.2cm} \faicon{circle} \faicon{circle} \faicon{circle-thin}\\
		Anglais \hspace{0.2cm} \faicon{circle} \faicon{circle} \faicon{circle-thin}\\
		Microcontrôleurs (8 bits / ARM) \hspace{0.2cm} \faicon{circle} \faicon{circle-thin} \faicon{circle-thin}\\
		Assembleur CISC Freescale \hspace{0.2cm} \faicon{circle} \faicon{circle-thin} \faicon{circle-thin}\\
		Réseau / infra \hspace{0.2cm} \faicon{circle} \faicon{circle-thin} \faicon{circle-thin}\\
		VHDL / SoPC \hspace{0.2cm} \faicon{circle} \faicon{circle-thin} \faicon{circle-thin}\\
		Python \hspace{0.2cm} \faicon{circle} \faicon{circle-thin} \faicon{circle-thin}\\
		\end{minipage}\if\debug1} \fi
	\if\debug1\fbox{\fi\begin{minipage}[H]{0.02\paperwidth}
		\ %alignement box
		\end{minipage}\if\debug1} \fi


	\vspace{0.5cm}%%%%%%%%%%%%%%%%%%%%%%%%%%%%%%%%%%%%%%%%%%%%%%%%%%%%FORMATIONS
	%\fbox{\textbf{FORMATIONS}}\\
	\fbox{\parbox[c][10pt][b]{67pt}{\centering\textbf{FORMATIONS}}}\\
	\vspace{0.5cm}
	\if\debug1\fbox{\fi\begin{minipage}[H]{0.30\paperwidth}
		\raggedleft
		\centering
		\textbf{2017 - 2019\\Master systèmes embarqués}\\
		ESTEI / Ynov - Bordeaux\\
		\ \\
		\textbf{2016 - 2017\\Bachelor systèmes\\embarqués \& robotique}\\
		ESTEI - Bordeaux\\
		\ \\
		\textbf{2014 - 2016\\BTS systèmes numériques\\électronique \& communication}\\
		Lycée Jean-Baptiste de Baudre - Agen\\
		\ \\
		\textbf{2012 - 2014\\BAC S - SVT\\Spécialité Informatique\\\& science du numérique}\\
		Lycée Bernard Palissy - Agen\\
		\end{minipage}\if\debug1} \fi

	\vspace{0.5cm}%%%%%%%%%%%%%%%%%%%%%%%%%%%%%%%%%%%%%%%%%%%%%%%%%%%%%%%%PROFIL
	%\fbox{\textbf{PROFIL}}\\
	\fbox{\parbox[c][10pt][b]{35pt}{\centering\textbf{PROFIL}}}\\
	\vspace{0.5cm}
	\if\debug1\fbox{\fi\begin{minipage}[H]{0.30\paperwidth}
		\raggedleft
		\centering
		\textbf{Caractère}\\
		Déterminé, perfectionniste, observateur\\
		\ \\
		\textbf{Centres d'intérêt}\\
		Musique, arts, logiciels libres, cuisine\\
		\end{minipage}\if\debug1} \fi
	\end{minipage}\if\debug1} \fi
\if\debug1\fbox{\fi\begin{minipage}[t]{0.6\paperwidth}%%%%%%%%%%%EXPÉRIENCES PRO
	\centering
	%\fbox{\textbf{EXPÉRIENCES PRO}}\\
	\fbox{\parbox[c][10pt][b]{91pt}{\centering\textbf{EXPÉRIENCES PRO}}}\\
	\vspace{0.5cm}
	\if\debug1\fbox{\fi\begin{minipage}[H]{0.58\paperwidth}
		\raggedright
		\textbf{Juin 2019 - Février 2020}\\
		\hspace{1cm} \textbf{Stage \& CDI - DMIC / Ubiwan, Bordeaux}\\
		\hspace{1cm} Développeur back-end (C++)\\
		\hspace{2cm} - Développement d'un framework de mesure de performances\\
		\hspace{2cm} - Étude de faisabilité d'une solution nouvelle\\
		\hspace{2cm} - Séparation d'un programme : librairie interne / partie métier\\
		\hspace{2cm} - Profiling (Hotspot, Valgrind, Heaptrack), débug, optimisation\\
		\hspace{2cm} - Migration d'anciens programmes Delphi vers C++\\
		\ \\
		\textbf{Mai 2015 - Juin 2015 (6 semaines)}\\
		\hspace{1cm} \textbf{Stage - Pesage47, Agen}\\
		\hspace{1cm} Technicien\\
		\hspace{2cm} - Diagnostic et réparation de balances / bascules\\
		\hspace{2cm} - Calibration / vérification de balances\\
		\end{minipage}\if\debug1} \fi

	\vspace{0.5cm}%%%%%%%%%%%%%%%%%%%%%%%%%%%%%%%%%%%%%%%%%%%%%%%%%PROJETS PERSO
	%\fbox{\textbf{PROJETS PERSO}}\\
	\fbox{\parbox[c][10pt][b]{79pt}{\centering\textbf{PROJETS PERSO}}}\\
	\vspace{0.5cm}
	\if\debug1\fbox{\fi\begin{minipage}[H]{0.58\paperwidth}
		\raggedright
		\href{https://github.com/thomaslepoix/Qucs-RFlayout}{
		\textbf{Octobre 2018 - Présent}\\
%		\hspace{1cm} \textbf{Contribution au logiciel libre de simulation électronique Qucs}\\
%		\hspace{1cm} Développement d'une interface d'export de typons hyperfréquences\\
%		\hspace{2cm} - Parser de données textes\\
%		\hspace{2cm} - Algorithme de reconstitution géométrique du typon\\
%		\hspace{2cm} - Export vers un logiciel de CAO (KiCad, Pcb-rnd, gEDA)\\
%		\hspace{2cm} - Interface graphique\\
%		\hspace{2cm} - Travail en C++ / Qt\\
%		\hspace{2cm} - Intégration au projet Qucs (Github)\\
%
		\hspace{1cm} \textbf{Amélioration de l'environnement libre de production hardware RF}\\
		\hspace{1cm} Objectif : Une alternative aux logiciels propriétaires et onéreux\\
		\hspace{2cm} - Création du logiciel libre Qucs-RFlayout\\
		\hspace{2cm} - Co-création du projet Open-RFlab\\
		\hspace{2cm} - Contribution aux simulateurs électroniques Qucs et OpenEMS\\
		\hspace{2cm} - Fédération d'une communauté\\
		\hspace{2cm} - Travail en C++ / Qt et en anglais\\
		\hspace{2cm} - Roadmap en collaboration avec la communauté\\
		}
		\ \\
%		\textbf{Novembre 2017 - Mai 2018}\\
%		\hspace{1cm} \textbf{Travail sur les solutions opensources en hyperfréquences}\\
%		\hspace{2cm} - Simulation / production de filtres microstrip \& antennes patch\\
%		\hspace{2cm} - Travail de recherche sur les antennes patch\\
%		\hspace{2cm} - Mise au point d'une méthode de design d'antenne patch\\
%		\hspace{2cm} - Utilisation des logiciels libres Qucs et OpenEMS\\
%		\hspace{2cm} - Rédaction de documentation sur les logiciels\\
%		\hspace{2cm} - Développement de scripts d'interfaçage de Qucs et KiCad\\
%
		\href{https://github.com/Dauliac/raisin}{
		\textbf{Mai 2020 - Présent}\\
		\hspace{1cm} \textbf{Raisin : Générateur de graphe CFG à partir de code C++}\\
		\hspace{1cm} Initiative et co-création du projet\\
		\hspace{2cm} - Utilisation des librairies LLVM/Clang\\
		\hspace{2cm} - Travail au niveau du compilateur sur la syntaxe du langage C++\\
		\hspace{2cm} - Travail en C++, OOP, design patterns, peer programming\\
		\hspace{2cm} - Projet naissant\\
		\hspace{2cm} - Aucun équivalent libre actuellement\\
		}
		\end{minipage}\if\debug1} \fi

	\vspace{0.5cm}%%%%%%%%%%%%%%%%%%%%%%%%%%%%%%%%%%%%%%%%%%%%%%PROJETS EN ÉCOLE
	%\fbox{\textbf{PROJETS EN ÉCOLE}}\\
	\fbox{\parbox[c][10pt][b]{94pt}{\centering\textbf{PROJETS EN ÉCOLE}}}\\
	\vspace{0.5cm}
	\if\debug1\fbox{\fi\begin{minipage}[H]{0.58\paperwidth}
		\raggedright
		\href{https://github.com/thibaudledo/Autoscope/blob/latex/Autoscope_6.pdf}{
		\textbf{Novembre 2018 - Juin 2019}\\
		\hspace{1cm} \textbf{Réalisation en groupe d'un télescope robotisé imprimé en 3D}\\
		\hspace{2cm} - Architecture système\\
		\hspace{2cm} - Coordination de l'équipe\\
%		\hspace{2cm} - Réalisation intégrale du hardware (basé sur Raspberry Pi)\\
		\hspace{2cm} - Réalisation intégrale du hardware (pluggable sur Raspberry Pi)\\
		\hspace{2cm} - Réalisation intégrale de l'OS (Yocto)\\
		\hspace{2cm} - Développement d'un driver GPS en C\\
		}
		\ \\
%		\textbf{Octobre 2017 - Février 2018}\\
%		\hspace{1cm} \textbf{Réalisation en groupe d'un véhicule robot télécommandé}\\
%		\hspace{1cm} Réalisation d'une télécommande Wifi avec Raspberry Pi\\
%		\hspace{2cm} - Carte d'IHM pluggable sur la Raspberry Pi\\
%		\hspace{2cm} - OS Linux sur mesure (Yocto)\\
%		\hspace{2cm} - Intégration d'un programme développé par d'autres\\
%		\ \\
		\textbf{Mars 2017 - Juin 2017}\\
		\hspace{1cm} \textbf{Réalisation en groupe d'un aéroglisseur télécommandé}\\
		\hspace{1cm} Réalisation intégrale du hardware\\
		\hspace{2cm} - Définition du cahier des charges\\
		\hspace{2cm} - Design du châssis\\
		\hspace{2cm} - Carte d'alimentation\\
		\hspace{2cm} - Carte microcontrôleur\\
		\hspace{2cm} - Module radio\\
		\hspace{2cm} - Module de programmation USB\\
		\end{minipage}\if\debug1} \fi
	\end{minipage}\if\debug1} \fi

\end{document}
